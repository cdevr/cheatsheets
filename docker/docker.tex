\documentclass[a4paper,landscape]{article}
\usepackage[utf8]{inputenc}
\usepackage{multicol}
\usepackage{geometry}
\usepackage{fancyhdr}
\usepackage{titlesec}

\geometry{left=2cm, right=2cm, top=2cm, bottom=2cm}

\setlength{\columnsep}{1cm}

\titleformat{\section}{\large\bfseries}{\thesection}{1em}{}
\titlespacing*{\section}{0pt}{1em}{0.5em}

\begin{document}

\begin{center}
    \LARGE\textbf{Docker Cheatsheet}
\end{center}

\begin{multicols}{3}

\section*{Docker Concepts}
\begin{itemize}
    \item \textbf{docker}: Creates, deploys and manages containers on different operating environmens (win/mac/linux/kubernetes)
    \item \textbf{image}: A filesystem containing the application docker will run.
    \item \textbf{layer}: The image storage format is composed of read-only file system layers; the container creates writable layers.
    \item \textbf{container}: Isolated environment within a kernel for an application to run. Has it's own root fs, networking, ...
    \item \textbf{Dockerfile}: A file detailing the steps to build a Docker image.
    \item \textbf{docker engine}: Docker installation running on a given host that can run containers, build images, ...
    \item \textbf{docker client}: A client that talks to local or remote Docker daemon which is used to manage containers, images, ...
    \item \textbf{docker daemon}: Background service managing images and containers.
    \item \textbf{docker registry}: Stores versioned docker images, associated with a tag, a string used to mark versions or variants of an image.
    \item \textbf{volume}: Directory shared between container and host.
\end{itemize}

\columnbreak

\section*{Docker cli}
\begin{verbatim}
docker run [OPTIONS] IMAGE[:TAG] [COMMAND]
\end{verbatim}

\begin{itemize}
    \item \texttt{--name=CONTAINER\_NAME}: Assign name to container.
    \item \texttt{--label NAME=VALUE}: Set metadata on container.
    \item \texttt{-d}, \texttt{--detach}: Run in the background.
    \item \texttt{-i}, \texttt{--interactive}: Interactive.
    \item \texttt{-t}, \texttt{--tty}: Allocate a pseudo-TTY.
    \item \texttt{--rm}: Automatically remove container when entrypoint process exits in the container.
    \item \texttt{-u}, \texttt{--user=[user[:group]]}: Run as specified user/uid, and group/gid.
    \item \texttt{--privileged}: Allow access to all host devices.
    \item \texttt{-v}, \texttt{--volume}: Bind mount a volume from the host.
    \item \texttt{--volume-from}: Mount volumes from another container.
    \item \texttt{-e}, \texttt{--env}: Set environment variables.
    \item \texttt{--env-file}: Read environment variables from a file.
    \item \texttt{-p}, \texttt{--publish [host\_port]:[container\_port]}: Publish container's ports to the host.
    \item \texttt{--expose}: Expose a port or a range of ports.
    \item \texttt{-h}, \texttt{--hostname}: Set container's hostname.
    \item \texttt{--network}: Connect container to a network.
    \item \texttt{--add-host}: Add a custom host-to-IP mapping.
    \item \texttt{--read-only}: Mount root read-only.
\end{itemize}

\section*{Dockerfile Instructions}
\textbf{Instructions:}
\begin{itemize}
    \item \textbf{FROM}: Set the base image to build from.
    \item \textbf{RUN}: Execute a command in the shell or exec form.
    \item \textbf{MAINTAINER}: Set the author field.
    \item \textbf{LABEL}: Add metadata.
    \item \textbf{ARG}: Define build-time variables.
    \item \textbf{ENV}: Set environment variables.
    \item \textbf{COPY}: Copy files + directories from host to image.
    \item \textbf{ADD}: Copy files from source (can be remote or archive) and add them to the image. \textit{discouraged}
    \item \textbf{VOLUME}: specify mount point for a volume.
    \item \textbf{EXPOSE}: Expose a port or ports.
    \item \textbf{WORKDIR}: Set the working directory for any \texttt{RUN}, \texttt{CMD}, \texttt{ENTRYPOINT}, \texttt{COPY}, and \texttt{ADD} instructions that follow it.
    \item \textbf{USER}: Set the user name or UID to use when running the image.
    \item \textbf{CMD}: Provide defaults for an executing container.
    \item \textbf{ENTRYPOINT}: Configure a container that will run as an executable.
\end{itemize}

\end{multicols}

\end{document}

